% ----------------------------------------------------------------
% Project Proposal
% Group Name: Blackout
% Group participants names: Thierry Backes, Sichen Li, Peng Zhou
% ----------------------------------------------------------------

%\documentclass[
%journal=ancac3, % for ACS Nano
%journal=acbcct, % for ACS Chem. Biol.
%journal=jacsat, % for undefined journal
%manuscript=article]{achemso}

\documentclass[11pt, a4paper]{article}

\usepackage{hyperref}
\usepackage{cite}
\usepackage[a4paper,bindingoffset=0.2in,%
            left=1in,right=1in,top=1in,bottom=1in,%
            footskip=.25in]{geometry}
       
\begin{document}
\title{Analyzing resilience of interdependent networks}
\author{Thierry Backes, Sichen Li, Peng Zhou}
\date{}
\maketitle

\section{Introduction}
With the trend of globalisation and liberalisation of the energy market, the combination of multiple energy systems faces big challenges. Understanding the survivability and resilience of interdependent networks, especially power grids, is an essential part in construction and maintenance.

We are analysing the propagation of failures in generated complex networks and in empirical data sets. We rely on the  SFINA package\footnote{Available at \url{https://github.com/SFINA}} as a Java framework for simulations on power grids. We want to understand how we can create more stable interconnected networks, and how the current networks can be changed to be more failure resistant while keeping the construction costs at a minimum. 

Several sources, such as~\cite{schneider2013towards}, study fully interconnected subnetworks and give us an understanding how those systems work, and what improvements one can take in order to improve the network resilience.

\section{The Model}

Our project relies on a network based model, which abstracts the power grid as a graph, where the nodes and edges represent generators, transformers, substation and the transmission lines between the stations. This abstraction makes it possible to include other networks, such as communication\footnote{ Such as the Italian blackout in 2003}  networks into our model.

We simulate failures, maintenance, accidents or attacks on the network by disconnecting selected parts. With the SFINA framework, we can take our abstraction, bind our graph to topological data, and reapply electrical flow to understand the severity of the generated failures. 

Our model strives to find the best method to restructure the networks in order to prohibit cascading failures and keeping the network functional for as long as possible. 
\section{Fundamental Questions}
\begin{enumerate}
\item How to measure and understand the robustness of the current infrastructure?
\item What are the key components resulting in cascading failure of power grids?
\item How can we improve the stability and functionality of power grids under partial failure?
\end{enumerate}

\section{Expected Results}
\begin{enumerate}
\item Understand and simulate historical blackouts.
\item Models to improve established networks and generate more reliable networks.
\item Models to connect existing subnetworks without damaging the integrity of the whole network
\end{enumerate}


\nocite{*}
\bibliographystyle{unsrt}
\bibliography{reference}

\section*{Acknowledgement}
We would like to give our special thanks to Dr. Evangelos Pournaras and Dr. Olivia Woolley Meza for their outstanding support.


\end{document}
